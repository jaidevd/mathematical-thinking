\documentclass[a4paper,12pt]{article} % This dfines the style of your paper
\usepackage[top = 2.5cm, bottom = 2.5cm, left = 2.5cm, right = 2.5cm]{geometry}
\usepackage[T1]{fontenc}
\usepackage[utf8]{inputenc}
\usepackage{multirow} % Multirow is for tables with multiple rows within one cell.
\usepackage{booktabs} % For even nicer tables.
\usepackage{graphicx}
\usepackage{setspace}
\usepackage{amsfonts}
\usepackage{amssymb}
\usepackage{amsmath}
\setlength{\parindent}{0in}
\usepackage{float}
\usepackage{fancyhdr}
\pagestyle{fancy} % With this command we can customize the header style.
\fancyhf{} % This makes sure we do not have other information in our header or footer.

\lhead{\footnotesize Graded Assignment: Week 10}% \lhead puts text in the top left corner. \footnotesize sets our font to a smaller size.

\rhead{\footnotesize Jaidev Deshpande (Roll No: 21F1003751)} %<---- Fill in your lastnames.
\cfoot{\footnotesize \thepage}

\begin{document}

\thispagestyle{empty} % This command disables the header on the first page.

\begin{tabular}{p{15.5cm}} % This is a simple tabular environment to align your text nicely
{\large \bf Mathematical Thinking} \\
BS In Data Science \& Applications  \\ Sept 2023 Term  \\ IIT Madras\\
\hline % \hline produces horizontal lines.
\\
\end{tabular} % Our tabular environment ends here.

\vspace*{0.3cm} % Now we want to add some vertical space in between the line and our title.

\begin{center} % Everything within the center environment is centered.
	{\Large \bf Graded Assignment: Week 10} % <---- Don't forget to put in the right number
	\vspace{2mm}

        % YOUR NAMES GO HERE
	{\bf Jaidev Deshpande (Roll No: 21F1003751)} % <---- Fill in your names here!

\end{center}

\vspace{0.4cm}

\begin{enumerate}
  \item \[\lim_{x\to 9} \frac{12 \sin(\sqrt{x} - 3)}{x - 9} = 12 \lim_{x\to 9} \frac{\sin(\sqrt{x} - 3)}{x - 9} \]
    $$ = 12 \lim_{x\to 9} \frac{\sin(\sqrt{x} - 3)}{(\sqrt{x} - 3)(\sqrt{x} + 3)}$$
    The limit can be factorized as:
    $$ = 12 \Big[\lim_{x\to 9} \frac{\sin(\sqrt{x} - 3)}{(\sqrt{x} - 3)} \times \lim_{x\to 9} \frac{1}{(\sqrt{x} + 3)} \Big]$$
    (Since both the limits in the factorization exist, the factorization is valid). The limit is, therefore,
    $$ 12 \Big[1 \times \frac{1}{6}\Big] = 2$$
    $$ \therefore \lim_{x\to 9} \frac{12 \sin(\sqrt{x} - 3)}{x - 9} = 2$$
  \item
    \begin{enumerate}
      \item \textbf{Incorrect.} Such a function can have more than one root, as long as the number of roots is odd.
      \item \textbf{Incorrect.} Such a function can have roots, as long as the number of roots is even.
      \item \textbf{Correct.} Suppose we have $f(x) = \cos{x} - \sin{x} - x - 2$. Then the roots of $f(x)$ are the solutions for the given equation. Evaluating $f(x)$ at the ends of the interval $\Big[\frac{-\pi}{2}, \frac{\pi}{2}\Big]$ gives us $\frac{\pi}{2} - 1$ and $-(\frac{\pi}{2} + 3)$ respectively. Since their signs are opposite, there must be at least one root in the interval.
      \item \textbf{Incorrect.} Counterexample: $sin(\frac{1}{x})$ is bounded between $[-1, 1]$ but is not defined at $x = 0$, and hence is not continuous.
    \end{enumerate}
  \item \[ \Phi = \frac{1 + \sqrt{5}}{2}, \phi = \frac{\sqrt{5} - 1}{2} \]
    \begin{enumerate}
      \item \textbf{Incorrect.} $\phi^2 \neq 1 - \Phi$.
      \item \textbf{Incorrect.} $\phi^2 \neq 1 + \phi$.
      \item \textbf{Incorrect.} $\phi^{20} \neq \frac{\phi^{21}}{\Phi}$
      \item \textbf{Correct.} $\phi^{20} = \frac{1 + \phi}{\Phi^{21}}$
    \end{enumerate}
  \item Suppose the limit exists. Let's call it $L$, i.e.
    $$L = \lim_{n \to \infty}\frac{t_{n+1}}{t_n}$$

    By the given recurrence relation,

    $$ t_{n+1} = 3t_n + 4t_{n - 1} $$
    $$ \therefore L = \lim_{n \to \infty}\frac{3t_n + 4t_{n - 1}}{t_n}$$
    $$ \therefore L = 3 + 4\lim_{n \to \infty}\frac{t_{n - 1}}{t_n}$$

    Now, note that the second term contains the reciprocal of $L$, i.e

    $$ \lim_{n \to \infty}\frac{t_{n - 1}}{t_n} = \frac{1}{L}$$
    $$ \therefore L = 3 + \frac{4}{L}$$
    $$ \therefore L^2 - 3L - 4 = 0$$

    The roots of this quadratic equation are $-1$ and $4$. Since the sequence contains only positive numbers, the ratio of two successive numbers, and therefore their limit cannot be negative.

    Thus, the limit is $L = 4$.

  \item Do later
  \item $f(x) = x^5 - 3x^3 + 1$
    Being a polynomial, $f(x)$ is continuous everywhere. Further, $f(0) = 1$ and $f(1) = -1$.
    Thus, the $f(x)$ must cross the X-axis at least once in the interval $[0, 1]$.
    Therefore, by IVT, there exists a root of $f(x)$ in the interval $[0, 1]$.
  \item $f, g: [0, 1] \rightarrow \mathbb{R}$ are continuous at $x_0$. Do later.
  \item
    \begin{enumerate}
      \item To show: $\Phi^2 = 1 + \Phi$\\
        LHS = $\Phi^2$
        $$ = \frac{1 + \sqrt{5}}{2} \times \frac{1 + \sqrt{5}}{2} $$
        $$ = \frac{6 + 2\sqrt{5}}{4} $$
        $$ = \frac{3 + \sqrt{5}}{2} $$
        RHS = $1 + \Phi$\\
        $$ = 1 + \frac{1 + \sqrt{5}}{2} $$
        $$ = \frac{3 + \sqrt{5}}{2} $$
        $$ \therefore \Phi^2 = 1 + \Phi$$
      \item To show: $\Phi^{n+1} = F_{n+1}\Phi + F_{n}$\\
        Proof by Induction:\\
        \begin{enumerate}
          \item Base case (n = 1)\\
            To show: $\Phi^{2} = F_{2}\Phi + F_{1}$\\
            Since $F_2 = F_1 = 1$, this amounts to showing that $\Phi^2 = \Phi + 1$\\.
            This is shown in the previous result.\\
            $\therefore$ base case is true.
          \item Inductive step: Assuming that $\Phi^{n+1} = F_{n+1}\Phi + F_{n}$\\
            Multiplying both sides by $\Phi$,\\
            $$\Phi^{n+2} = F_{n+1}\Phi^2 + F_{n}\Phi$$

            Now, by definition of the Fibonacci sequence, $F_{n}$ can be written as follows:
            $$ F_{n} = F_{n+2} - F_{n + 1}$$

            Substituting these values in the equation, we get:
            $$\Phi^{n+2} = F_{n+1}\Phi^2 + \Phi(F_{n+2} - F_{n+1})$$
            $$ = F_{n+1}\Phi^2 - F_{n+1}\Phi + \Phi F_{n+2} $$
            $$ = F_{n+1}(\Phi^2 - \Phi) + \Phi F_{n+2}$$
            We know from the previous result that $\Phi^2 - \Phi = 1$. Therefore,
            $$\Phi^{n+2} = F_{n+2}\Phi + F_{n+1}$$

            Thus, if
              $$\Phi^{n+1} = F_{n+1}\Phi + F_{n}$$
            then
              $$\Phi^{n+2} = F_{n+2}\Phi + F_{n+1}$$

            $\therefore$ by induction, the result holds $\forall n \in \mathbb{N}$.\\
            $\blacksquare$
        \end{enumerate}
    \end{enumerate}
\end{enumerate}
\end{document}
