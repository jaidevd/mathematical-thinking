\documentclass[a4paper,12pt]{article} % This defines the style of your paper
\usepackage[top = 2.5cm, bottom = 2.5cm, left = 2.5cm, right = 2.5cm]{geometry} 
\usepackage[T1]{fontenc}
\usepackage[utf8]{inputenc}
\usepackage{multirow} % Multirow is for tables with multiple rows within one cell.
\usepackage{booktabs} % For even nicer tables.
\usepackage{graphicx} 
\usepackage{setspace}
\usepackage{amsfonts}
\usepackage{amssymb}
\usepackage{amsmath}
\setlength{\parindent}{0in}
\usepackage{float}
\usepackage{fancyhdr}
\pagestyle{fancy} % With this command we can customize the header style.
\fancyhf{} % This makes sure we do not have other information in our header or footer.

\lhead{\footnotesize Graded Assignment: Week 1}% \lhead puts text in the top left corner. \footnotesize sets our font to a smaller size.

\rhead{\footnotesize Jaidev Deshpande (Roll No: 21F1003751)} %<---- Fill in your lastnames.
\cfoot{\footnotesize \thepage} 

\begin{document}

\thispagestyle{empty} % This command disables the header on the first page. 

\begin{tabular}{p{15.5cm}} % This is a simple tabular environment to align your text nicely 
{\large \bf Mathematical Thinking} \\
BS In Data Science \& Applications  \\ Sept 2023 Term  \\ IIT Madras\\
\hline % \hline produces horizontal lines.
\\
\end{tabular} % Our tabular environment ends here.

\vspace*{0.3cm} % Now we want to add some vertical space in between the line and our title.

\begin{center} % Everything within the center environment is centered.
	{\Large \bf Graded Assignment: Week 1} % <---- Don't forget to put in the right number
	\vspace{2mm}
	
        % YOUR NAMES GO HERE
	{\bf Jaidev Deshpande (Roll No: 21F1003751)} % <---- Fill in your names here!
		
\end{center}  

\vspace{0.4cm}

\begin{enumerate}

\item For a function \(f: X \rightarrow Y \), and \\
	\(A \subset B \subset X \), \\
	(c) \(f(A) \subset f(B)\) is always true.

\item For the series \(S = 1 + \frac{1}{2^3} + \frac{1}{3^3} \dots \),\\
	the following expressions denote the partial sum \(S_n\):
	\subitem (a) \[S_n = \sum_{k=2}^{n + 1} \frac{1}{(k - 1)^3}\]
	\subitem (b) \[S_n = \sum_{k=1}^{n} \frac{1}{k^3}\]
	\subitem (c) \[S_n = \sum_{k=0}^{n - 1} \frac{1}{(k + 1)^3}\]

\item For the sequence
	\[ 2, 2-\frac{1}{2}, 2 + \frac{1}{2-\frac{1}{2}}, 2 - \frac{1}{2 + \frac{1}{2-\frac{1}{2}}}, \dots \]\\
	the recursive formula is:
	\subitem (d) \[ a_0 = 2; a_n = 2 + \frac{(-1)^n}{a_{n - 1}} \forall n > 0 \]

\item[6] \(f : \mathbb{N} \cup \{0\} \rightarrow \mathbb{N} \cup \{0\} \) such that \(f(m + n) = f(m) + f(n) \forall m, n \in \mathbb{N} \)
	\subitem(a) To prove: \(f(0) = 0\)\\
	Suppose there exist two numbers \(p, q \in \mathbb{N} \cup \{0\}\) such that \(p + q = 0\).\\
	No two natural numbers can sum to 0.\\
		\( \therefore p \notin \mathbb{N} \) and \(q \notin \mathbb{N}\)\\
		\[\therefore p = q = 0\]
	By definition of \(f\), \(f(p + q) = f(p) + f(q)\)
	\[\therefore f(0) = f(0) + f(0)\]
	\[\therefore f(0) = 2f(0)\]
	This can only happen if \(f(0) = 0\). $ \blacksquare $

	\subitem(b) To prove: \(f(n) = nf(1) \forall n \in \mathbb{N}\)
	Let us denote this proposition as \(P(n)\), i.e.
		\[P(n) := f(n) = nf(1)\]
	We shall prove this by induction.
	Establish the base case \(P(1)\), i.e.
		\[f(1) = 1f(1)\]
	Now, in the induction step, we need to show that if \(P(n)\) is true, then \(P(n + 1)\) is also true.\\
	To begin, we suppose that \(P(n)\) is true, i.e.
		\[f(n) = nf(1)\]
	Then, by the definition of \(f, f(n + 1)\) can be written as:
		\[f(n + 1) = f(n) + f(1)\]
		\[ \therefore f(n + 1) = nf(1) + f(1)\]
		\[\therefore f(n + 1) = f(1)(n + 1)\]
	This equation is nothing but the statement of $P(n + 1)$, by definition of $P(n)$.\\
	Thus, we establish that if $P(n)$ is true, then $P(n + 1)$ is also true.\\
	Therefore, by the principle of mathematical induction, $P(n)$ is true for all $n \in \mathbb{N}$. $ \blacksquare $


\item[7] To prove: \(x(y + z) = xy + xz\)
	By definition of multiplication of natural numbers, the LHS can be written as:
		\[x(y + z) = \underbrace{(y + z) + (y + z) \dots (y + z)}_{x   \text{ times}}\]
	By commutative property of addition of natural numbers, the RHS of the above equation can be rearranged as:
		\[x(y + z) = \underbrace{y + y \dots y}_{x \text{ times}} + \underbrace{z + z \dots z}_{x \text{ times}}\]
	which is then simplified as:
		\(x(y + z) = xy + xz\) $ \blacksquare $

\item[8.] Consider set $\mathbb{N}$ of all positive odd integers, such that $\forall n \in N, n+ = n + 2$.\\
	\subitem(i) Since $1 \in \mathbb{N}$, the first Peano axiom is satisfied.\\

	\subitem(ii) Since $n+$ is defined suitably for odd natural numbers, the second Peano axiom is satisfied.\\

	\subitem(iii) Two distinct odd numbers cannot have the same consequent odd number. Therefore, the third Peano axiom is satisfied.\\

	\subitem(iv) No positive odd number has 1 as its successor. Therefore, the fourth Peano axiom is satisfied.\\

	\subitem(v) We know that \(1 \in \mathbb{N}\) and also that if \(n \in \mathbb{N}\), then \(n+ \in \mathbb{N}\).\\
		Thus, the fifth Peano axiom is satisfied.\\
		\textbf{Caveat:} According to the fifth Peano axiom, the set $\mathbb{N}$\\
		should be same as the set of natural numbers, even if we have defined it as\\
		the set of odd natural numbers.\\
		Note that they have the same cardinality and there exists a bijection between them.\\
		But, ultimately, they are not the same set.
\end{enumerate}
\end{document}
