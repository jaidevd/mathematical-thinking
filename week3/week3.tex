\documentclass[a4paper,12pt]{article} % This dfines the style of your paper
\usepackage[top = 2.5cm, bottom = 2.5cm, left = 2.5cm, right = 2.5cm]{geometry} 
\usepackage[T1]{fontenc}
\usepackage[utf8]{inputenc}
\usepackage{multirow} % Multirow is for tables with multiple rows within one cell.
\usepackage{booktabs} % For even nicer tables.
\usepackage{graphicx} 
\usepackage{setspace}
\usepackage{amsfonts}
\usepackage{amssymb}
\usepackage{amsmath}
\setlength{\parindent}{0in}
\usepackage{float}
\usepackage{fancyhdr}
\pagestyle{fancy} % With this command we can customize the header style.
\fancyhf{} % This makes sure we do not have other information in our header or footer.

\lhead{\footnotesize Graded Assignment: Week 3}% \lhead puts text in the top left corner. \footnotesize sets our font to a smaller size.

\rhead{\footnotesize Jaidev Deshpande (Roll No: 21F1003751)} %<---- Fill in your lastnames.
\cfoot{\footnotesize \thepage} 

\begin{document}

\thispagestyle{empty} % This command disables the header on the first page. 

\begin{tabular}{p{15.5cm}} % This is a simple tabular environment to align your text nicely 
{\large \bf Mathematical Thinking} \\
BS In Data Science \& Applications  \\ Sept 2023 Term  \\ IIT Madras\\
\hline % \hline produces horizontal lines.
\\
\end{tabular} % Our tabular environment ends here.

\vspace*{0.3cm} % Now we want to add some vertical space in between the line and our title.

\begin{center} % Everything within the center environment is centered.
	{\Large \bf Graded Assignment: Week 3} % <---- Don't forget to put in the right number
	\vspace{2mm}
	
        % YOUR NAMES GO HERE
	{\bf Jaidev Deshpande (Roll No: 21F1003751)} % <---- Fill in your names here!
		
\end{center}  

\vspace{0.4cm}

\begin{enumerate}

\item Find the remainder when $2^{20} + 3^{30} + 4^{40} + 5^{50}$ is divided by 7.\\
	Let us denote the "modulo" operator by "\%", i.e. $a$ \% $b$ is the remainder when $a$ is divided by $b$.\\
	We will first break the sum into individual parts, and find the remainder when each part is divided by 7.\\
	\subitem Inspecting remainders obtained when powers of 2 are divided by 7, we get the following pattern:\\
		$2^{(1)}$ \% 7 = 2  \% 7 = 2\\
		$2^{(2)}$ \% 7 = 4  \% 7 = 4\\
		$2^{(3)}$ \% 7 = 8  \% 7 = 1\\
		$2^{(4)}$ \% 7 = 16 \% 7 = 2\\
		Thus, the pattern repeats after 3 steps. Since 20 \% 3 = 2, the second number in this pattern is the remainder when $2^{20}$ is divided by 7.\\
		Therefore, $2^{(20)}$ \% 7 = 4\\

	\subitem Inspecting remainders obtained when powers of 3 are divided by 7, we get the following pattern:\\
		$3^{(1)}$ \% 7 = 3 \% 7 = 3\\
		$3^{(2)}$ \% 7 = 9 \% 7 = 2\\
		$3^{(3)}$ \% 7 = 27 \% 7 = 6\\
		$3^{(4)}$ \% 7 = 81 \% 7 = 4\\
		$3^{(5)}$ \% 7 = 243 \% 7 = 5\\
		$3^{(6)}$ \% 7 = 729 \% 7 = 1\\
		$3^{(7)}$ \% 7 = 2187 \% 7 = 3\\
		The pattern repeats after 6 steps. Since 30 \% 6 = 0, the last number in this pattern is the remainder when $3^{30}$ is divided by 7.\\
		Therefore, $3^{(30)}$ \% 7 = 1\\

	\subitem Inspecting remainders obtained when powers of 4 are divided by 7, we get the following pattern:\\
		$4^{(1)}$ \% 7 = 4 \% 7 = 4\\
		$4^{(2)}$ \% 7 = 16 \% 7 = 2\\
		$4^{(3)}$ \% 7 = 64 \% 7 = 1\\
		$4^{(4)}$ \% 7 = 256 \% 7 = 4\\
		The pattern repeats after 3 steps. Since 40 \% 3 = 1, the first number in this pattern is the remainder when $4^{40}$ is divided by 7.\\
		Therefore, $4^{(40)}$ \% 7 = 4\\

	\subitem Inspecting remainders obtained when powers of 5 are divided by 7, we get the following pattern:\\
		$5^{(1)}$ \% 7 = 5 \% 7 = 5\\
		$5^{(2)}$ \% 7 = 25 \% 7 = 4\\
		$5^{(3)}$ \% 7 = 125 \% 7 = 6\\
		$5^{(4)}$ \% 7 = 625 \% 7 = 2\\
		$5^{(5)}$ \% 7 = 3125 \% 7 = 3\\
		$5^{(6)}$ \% 7 = 15625 \% 7 = 1\\
		$5^{(7)}$ \% 7 = 78125 \% 7 = 5\\
		The pattern repeats after 6 steps. Since 50 \% 6 = 2, the second number in this pattern is the remainder when $5^{50}$ is divided by 7.\\
		Therefore, $5^{(50)}$ \% 7 = 4\\
	
	Adding up the remainders from the four components of the sum, we get 4 + 4 + 4 + 1 = 13.\\
	Since 13 \% 7 = 6, the remainder when $2^{20} + 3^{30} + 4^{40} + 5^{50}$ is divided by 7 is 6.\\

\item For $a, b \in \mathbb{N}$, then the following are always true:\\
	\subitem A. If $a^3|b^3$, then $a|b$

\item If $m, n \in \mathbb{N}$ and gcd($m, n$) = 1, then the following options are always true:\\
	\subitem A. $\exists x, y \in \mathbb{Z}$ such that $mx - ny = 1$

\item The following equations have solutions for $a, b \in \mathbb{Z}$:\\
	\subitem C. $23a + 11b = 120$

\item The number $a$ having digits $a_1, a_2, a_3 \dots a_n$ can be written as follows:\\
	$$ a = 10^{n - 1} a_1 + 10^{n - 2} a_2 + 10^{n - 3} a_3 \dots + 10^3 a_{n - 3} $$
	$$ + 10^2 a_{n - 2} + 10^1 a_{n - 1} + a_n $$
	Note that the second row here represents the number formed by the last three digits of $a$. Let us denote the first and the second rows by $p$ and $q$ respectively, i.e.
		$$ p = 10^{n - 1} a_1 + 10^{n - 2} a_2 + 10^{n - 3} a_3 \dots + 10^3 a_{n - 3} $$
		$$ q = 10^2 a_{n - 2} + 10^1 a_{n - 1} + a_n $$
		$$\therefore a = p + q$$


	Taking $10^3$ as the common factor from $p$,

	$$ p = 10^3 (10^{n - 4} a_1 + 10^{n - 5} a_2 + 10^{n - 6} a_3 \dots + 10 a_{n - 4} + a_{n - 3}) $$

	$p$ is an integer mulitple of 1000, and hence divisible by 1000. Since 1000 is divisible by 8, $p$ is also divisible by 8.

	Now, we know that $8 | a$ and $8 | p$. Then we can declare two numbers $x, y \in \mathbb{Z}$ such that:
	$$ a = 8x $$
	$$ p = 8y $$
	
	Subtracting the second equation from the first, we get:
	$$ q = a - p = 8x - 8y = 8(x - y) $$

	Since $x, y \in \mathbb{Z}$, $x - y \in \mathbb{Z}$.\\
	$\therefore q$ is an integer multiple of 8.\\

	Thus, if any number having four or more digits is divisible by 8, then the number formed by its last three digits is also divisible by 8.\\
	$\blacksquare$\\

\item Show that if $n | m$ and $n | k$, then $n^2 | mk$.\\
	(Assuming, $m, n, k \in \mathbb{Z}$)\\
	Since $n | m$, we can write $m$ as\\
	$$ m = nx; x \in \mathbb{Z}$$
	Similarly, since $n | k$, we can write $k$ as\\
	$$ k = ny; y \in \mathbb{Z}$$
	Multiplying the two equations, we get:

	$$ mk = n^{2}xy$$

	Since $x, y \in \mathbb{Z}$, $xy \in \mathbb{Z}$. Let's denote $p = xy$. Then,

	$$mk = n^{2}p$$

	Now, independently, suppose that we get the remainder $r$ and quotient $q$ when dividing $mk$ by $n^{2}$.

	$$\therefore mk = n^{2}q + r$$

	The division algorithm states that $q$ and $r$ are \textit{unique}.

	Therefore, we can equate the two equations obtained for $mk$:

	$$n^{2}p = n^{2}q + r$$

	Because of uniqueness of the quotient and the remainder, the above equation holds iff $p = q$ and $r = 0$.

	$\therefore n^2$ divides $mk$.

	$\blacksquare$

\item Prove that gcd(gcd($m, n$), $k$) = gcd($m$, gcd($n, k$)).\\
	Suppose the set $F_x$ is the set of all factors of $x$. E.g: $F_{12} = \{1, 2, 3, 4, 6, 12\}$.\\
	Thus, consider three sets $F_m$, $F_n$ and $F_k$ of the factors of $m$, $n$ and $k$ respectively.\\
	Then, gcd($m, n$) can be written as:\\
	$$ gcd(m, n) = max(F_m \cap F_n) $$

	So, the LHS of the equation becomes:
	$$ gcd(gcd(m, n), k) = max(max(F_m \cap F_n) \cap F_k) $$

	Since $max$ is an associative operator, the LHS can be further simplified as,

	$$ gcd(gcd(m, n), k) = max(F_m \cap F_n \cap F_k)$$

	Similarly, the RHS can be written as:

	$$ gcd(m, gcd(n, k)) = gcd(m, max(F_n \cap F_k)) $$
	$$ \therefore gcd(m, gcd(n, k)) = max(F_m \cap F_n \cap F_k) $$

	which is the same as the LHS.\\
	$\blacksquare$

\item $r$ is the remainder when $m$ is divided by $n$. Suppose the quotient is $q$. Then, we can write $m$ as follows:
	$$ m = nq + r; q, r \in \mathbb{Z} $$

	Now, since $g = $ gcd($m ,n$), by Euclid's reduction algorithm, $g$ is also the gcd of $n$ and $m - nk$, where $k$ can be any integer. So, the following holds,

	$$ g = gcd(m - nq, n) $$

	But $m - nq$ is nothing but $r$, as specified above.

	$$\therefore g = gcd(r, n)$$

	Which is the same as $g'$.

	Therefore, $g = g' \implies g'|g$. $\blacksquare$

\end{enumerate}
\end{document}
