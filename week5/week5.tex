\documentclass[a4paper,12pt]{article} % This dfines the style of your paper
\usepackage[top = 2.5cm, bottom = 2.5cm, left = 2.5cm, right = 2.5cm]{geometry} 
\usepackage[T1]{fontenc}
\usepackage[utf8]{inputenc}
\usepackage{multirow} % Multirow is for tables with multiple rows within one cell.
\usepackage{booktabs} % For even nicer tables.
\usepackage{graphicx} 
\usepackage{setspace}
\usepackage{amsfonts}
\usepackage{amssymb}
\usepackage{amsmath}
\setlength{\parindent}{0in}
\usepackage{float}
\usepackage{fancyhdr}
\pagestyle{fancy} % With this command we can customize the header style.
\fancyhf{} % This makes sure we do not have other information in our header or footer.

\lhead{\footnotesize Graded Assignment: Week 5}% \lhead puts text in the top left corner. \footnotesize sets our font to a smaller size.

\rhead{\footnotesize Jaidev Deshpande (Roll No: 21F1003751)} %<---- Fill in your lastnames.
\cfoot{\footnotesize \thepage} 

\begin{document}

\thispagestyle{empty} % This command disables the header on the first page. 

\begin{tabular}{p{15.5cm}} % This is a simple tabular environment to align your text nicely 
{\large \bf Mathematical Thinking} \\
BS In Data Science \& Applications  \\ Sept 2023 Term  \\ IIT Madras\\
\hline % \hline produces horizontal lines.
\\
\end{tabular} % Our tabular environment ends here.

\vspace*{0.3cm} % Now we want to add some vertical space in between the line and our title.

\begin{center} % Everything within the center environment is centered.
	{\Large \bf Graded Assignment: Week 5} % <---- Don't forget to put in the right number
	\vspace{2mm}
	
        % YOUR NAMES GO HERE
	{\bf Jaidev Deshpande (Roll No: 21F1003751)} % <---- Fill in your names here!
		
\end{center}  

\vspace{0.4cm}

\begin{enumerate}

\item We are essentially looking for primes <= 200 such that their last digit is 7. These are\\
	\{7, 17, 37, 47, 67, 97, 107, 127, 137, 157, 167, 197\}\\
	Thus, there are \textbf{12} such primes.

\item The following statements are true:
	\begin{enumerate}
		\item[2.] Out of any 13 integers, there exist 4 whose sum is a multiple of 4.
		\item[3.] Out of any 10 integers, there exist 4 whose sum is a multiple of 4.
		\item[4.] Out of any 11 integers, there exist 4 whose sum is a multiple of 4.
	\end{enumerate}
\item Suppose $S$ is the set of all possible arrangements, and suppose $A, B, C, D$ as the events described in the hint, we are interested in finding the following:
	$$ |S| - |A \cup B \cup C \cup D| $$
	By the Inclusion-Exclusion principle,\\
		$$ |A \cup B \cup C \cup D| = |A| + |B| + |C| + |D| $$
		$$ - |A \cap B| - |A \cap C| - |A \cap D| - |B \cap C| - |B \cap D| - |C \cap D| $$
		$$ + |A \cap B \cap C| + |A \cap B \cap D| + |A \cap C \cap D| + |B \cap C \cap D| $$
		$$ - |A \cap B \cap C \cap D| $$
	Computing the individual terms,
	\begin{itemize}
		\item $|A| = |B| = |C| = |D| = 24$
		\item $|A \cap B| = 6$
		\item $|A \cap C| = 2$
		\item $|A \cap D| = 2$
		\item $|B \cap C| = 6$
		\item $|B \cap D| = 2$
		\item $|C \cap D| = 6$
		\item $|A \cap B \cap C| = 2$
		\item $|A \cap B \cap D| = 1$
		\item $|A \cap C \cap D| = 1$
		\item $|B \cap C \cap D| = 2$
		\item $|A \cap B \cap C \cap D| = 1$
	\end{itemize}
	$$\therefore |A \cup B \cup C \cup D| = 4 \times 24 $$
	$$ - 6 - 2 - 2 - 6 - 2 - 6 $$
	$$ + 2 + 1 + 1 + 2 $$
	$$ - 1  = 79$$

	Thus, the desired number is $|S| - 79 = 6! - 79 = 641$.

\item Let us first find numbers <= 3001 which are multiples of 7 and 8, i.e. multiples of 56. These will be of the form $56k$ such that $k>=1$ and $56k <= 3001$.
	$$ \Bigl\lfloor \frac{3001}{56} \Bigr\rfloor = 53.  $$
	Thus, there are 53 numbers <= 3001 which are multiples of 7 and 8.
	Finally, the answer is $3001 - 53 = 2948$.

\item Suppose we group the elements of $S$ based on the remainders they leave when divided by 10. Thus, 9 such groups are possible, since no element is a multiple of 10.\\
	This is equivalent to putting 10 items into 9 boxes.\\
	Then, by the pigeonhole principle, there must exist a box with at least 2 elements.\\
	In other words, there are at least two elements in $S$ such that both leave the same remainder when divided by 10.\\
	Let $a$ and $b$ be these two integers.\\
	Then, by the division algorithm,
	$$ a = 10p + r $$
	$$ b = 10q + r $$
	where $p, q \in \mathbb{Z}$ and $r \in \{1, 2, \dots 9\}$.\\
	
	Subtracting the second equation from the first, we get

	$$ a - b = 10(p - q) $$

	Since $p, q \in \mathbb{Z}$, $p - q \in \mathbb{Z}$.\\
	Thus, $a - b$ and $b - a$ are both integer multiples of 10.\\
	Therefore, it is always possible to pick two elements from $S$ such that their difference is a multiple of 10.\\
	$\blacksquare$

\item The midpoint of the line segment joining two points $(a, b)$ and $(c, d)$ is

	$$\Big(\frac{a + c}{2}, \frac{b + d}{2}\Big)$$
	Note that these two numbers can be integers only if $a + c$ and $b + d$ are both even.\\
	This, in turn can happen only when both the following conditions are true:

	\begin{itemize}
		\item $a$ and $c$ are both odd or both even.
		\item $b$ and $d$ are both odd or both even.
	\end{itemize}

	Now, consider that any pair $(x, y)$ such that $x, y \in \mathbb{Z}$ can be grouped into one of the following four classes:

	\begin{itemize}
		\item $x$ and $y$ are both even.
		\item $x$ is even and $y$ is odd.
		\item $x$ is odd $y$ is even.
		\item $x$ and $y$ are both odd.
	\end{itemize}

	If we have 5 integer pairs (or points on an XY plane) grouped into these 4 classes, then by the pigeonhole principle, at least two of them must belong to the same class.\\
	Let's call these two points $(x_1, y_1)$ and $(x_2, y_2)$.\\
	As a result, one of the following cases must be true:

	\begin{enumerate}
		\item $x_1, y_1, x_2, y_2$ are all even.
		\item $x_1, x_2$ are even, and $y_1, y_2$ are odd.
		\item $x_1, x_2$ are odd, and $y_1, y_2$ are even.
		\item $x_1, y_1, x_2, y_2$ are all odd.
	\end{enumerate}
	Note that all four cases satisfy both the conditions mentioned above, which are necessary for having a midpoint which has integer coordinates.\\
	Theferore, given five points on an XY plane with integer coordinates, there always exists a pair whose midpoint has integer coordinates.\\
	$\blacksquare$

\item $S \subset \{1, 2, 3, \dots, 10\}$ and $|S| = 7$
	\begin{enumerate}
		\item Suppose we arrange the elements of $S$ in a table as follows:
			\begin{center}
			\begin{tabular}{ c c }
			 1 & 10 \\ 
			 2 & 9 \\  
			 3 & 8 \\
			 4 & 7 \\
			 5 & 6 \\
			\end{tabular}
			\end{center}
			Note that each row sums to 11. Now, suppose we view each column as a pigeonhole.\\
			Then, by the pigeonhole principle, in order to pick 7 elements from $S$, we must pick all elements from one column, and at exactly two elements from the other column.\\
			Suppose we pick all from the first column, i.e. $\{1, 2, 3, 4, 5\}$. Then, both the elements picked from the second column (in order to complete the set), will pair with some two elements out of $\{1, 2, 3, 4, 5\}$, thus producing at least two pairs which some to 11.\\
			Similarly, it can be shown that if all elements in the second column are picked, the two elements needed from the first column will also generate two pairs of numbers which sum to 11.\\
			Therefore, it is always possible to pick 7 elements from $S$ such that at least two pairs of them sum to 11.\\
			$\blacksquare$
		\item The conclusion above does not hold if $|S| = 6$. The best we can do with $|S| = 6$ is assert that there will exist at least 1 pair which sums to 11.\\
			Counter-example: $S = \{1, 2, 3, 5, 6, 7\}$ - contains only one pair that adds to 11.
	\end{enumerate}

\end{enumerate}
\end{document}
