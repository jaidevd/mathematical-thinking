\documentclass[a4paper,12pt]{article} % This dfines the style of your paper
\usepackage[top = 2.5cm, bottom = 2.5cm, left = 2.5cm, right = 2.5cm]{geometry}
\usepackage[T1]{fontenc}
\usepackage[utf8]{inputenc}
\usepackage{multirow} % Multirow is for tables with multiple rows within one cell.
\usepackage{booktabs} % For even nicer tables.
\usepackage{graphicx}
\usepackage{setspace}
\usepackage{amsfonts}
\usepackage{amssymb}
\usepackage{amsmath}
\setlength{\parindent}{0in}
\usepackage{float}
\usepackage{fancyhdr}
\pagestyle{fancy} % With this command we can customize the header style.
\fancyhf{} % This makes sure we do not have other information in our header or footer.

\lhead{\footnotesize Graded Assignment: Week 11}% \lhead puts text in the top left corner. \footnotesize sets our font to a smaller size.

\rhead{\footnotesize Jaidev Deshpande (Roll No: 21F1003751)} %<---- Fill in your lastnames.
\cfoot{\footnotesize \thepage}

\begin{document}

\thispagestyle{empty} % This command disables the header on the first page.

\begin{tabular}{p{15.5cm}} % This is a simple tabular environment to align your text nicely
{\large \bf Mathematical Thinking} \\
BS In Data Science \& Applications  \\ Sept 2023 Term  \\ IIT Madras\\
\hline % \hline produces horizontal lines.
\\
\end{tabular} % Our tabular environment ends here.

\vspace*{0.3cm} % Now we want to add some vertical space in between the line and our title.

\begin{center} % Everything within the center environment is centered.
	{\Large \bf Graded Assignment: Week 11} % <---- Don't forget to put in the right number
	\vspace{2mm}

        % YOUR NAMES GO HERE
	{\bf Jaidev Deshpande (Roll No: 21F1003751)} % <---- Fill in your names here!

\end{center}

\vspace{0.4cm}

\begin{enumerate}
  \item The continued fraction for $\sqrt{5}$ is:\\
      (b) $$ 2 + \frac{1}{2 + \frac{1}{2 + \frac{1}{\ldots}}}$$
  \item The number of ways of partitioning a set of 10 elements into 5 blocks, each of size 2, is:
    $$ \frac{\binom{10}{2}\binom{8}{2}\binom{6}{2}\binom{4}{2}}{5!}$$
    $$ = 945 $$
  \item The area of the rectangle is $xy$. The fencing is equal to $2x + y = 200$.\\
    Thus, $y = 200 - 2x$.\\
    The area then becomes

    $$ f(x) = x(200 - 2x) $$
    $$      = -2x^2 + 200x $$

    The maximum area can be found by solving $f'(x) = 0$.\\
    $$ f'(x) = -4x + 200 $$
    $$ \therefore argmax_{x}f(x) = 50 $$

    The corresponding value of $y$ is $100$.\\
    Thus the maximum possible area is $5000$.
  \item $h(t) = -25t^2 + 144$\\

    (Assuming KMS system of measurement)

    Suppose the rock hits the ground at tim $T$. Then, $h(T) = 0$ meters.\\
    Solving for $T$,\\
    $$ 25T^2 = 144$$
    $$ \therefore T = \frac{12}{5} s$$

    The average velocity can then computed as:

    $$ v_{avg} = \frac{h(T) - h(0)}{T} $$
    $$         = \frac{-144}{\frac{12}{5}} $$
    $$         = -60 m/s$$
    (Shown for average \textit{velocity} instead of average \textit{speed})
  \item Given the equation for $h(t)$, the instantaneous velocity can be obtained by:

    $$ v(t) = \frac{dh(t)}{dt} $$
    $$ \therefore v(t)      = -50t $$

    Thus, if $v(t) = -60$,
    $$ -50t = -60 $$
    $$ \therefore t = \frac{6}{5} s$$
    The rock matches it's average velocity at $t = 1.2$ seconds.
  \item $S(1, \sqrt{2}) = \{a + b\sqrt{2}|a, b \in \mathbb{R}\}$
    \begin{enumerate}
      \item To show: $S$ is closed under addition.
        Suppose there are two numbers $p, q \in S$ such that
        $$p = a_1 + b_1\sqrt{2}$$
        $$p = a_2 + b_2\sqrt{2}$$

        Adding them, we get:
        $$ p + q = (a_1 + a_2) + (b_1 + b_2)\sqrt{2}$$
        Since $\mathbb{R}$ is closed under addition, the sum of any two real numbers is a real number.
        Thus, $a_1 + a_2$ and $b_1 + b_2$ are both real numbers. Let's call them $A$ and $B$ respectively.

        $$ \therefore p + q = A + B\sqrt{2}; A, B \in \mathbb{R}$$
        $$ \therefore p + q \in S$$
        $\therefore S$ is closed under addition.\\
        $\blacksquare$
      \item To show: $S$ is closed under multiplication.
        Suppose there are two numbers $p, q \in S$ such that
        $$p = a_1 + b_1\sqrt{2}$$
        $$p = a_2 + b_2\sqrt{2}$$

        Multiplying them, we get:
        $$pq = (a_1 + b_1\sqrt{2})(a_2 + b_2\sqrt{2})$$
        $$\therefore pq = a_1 a_2 + a_1 b_2\sqrt{2} + a_2 b_1\sqrt{2} + 2b_1b_2$$
        $$\therefore pq = (a_1a_2 + 2b_1b_2) + (a_1b_2 + a_2b_1)\sqrt{2}$$

        Since $\mathbb{R}$ is closed under addition and multiplication, $a_1a_2 + 2b_1b_2$ and $a_1b_2 + a_2b_1$ are both real numbers. Let's call them $A$ and $B$ respectively.

        $$\therefore pq = A + B\sqrt{2}; A, B \in \mathbb{R}$$
        Thus, $pq \in S$.\\
        $\therefore S$ is closed under multiplication.\\
        $\blacksquare$
    \end{enumerate}
  \item \textit{Notation}: $f^{n}(x)$ denotes the $n^{th}$ derivative of the function $f(x)$.\\
    To show:
    $$ (f(x) g(x))^n = \sum_{k=0}^{n} \binom{n}{k} f^{n-k}(x)g^{k}(x) $$

    Proof by Induction:\\
    \begin{enumerate}
      \item Base case: $n = 1$
        $$ (f(x) g(x))^1 = \sum_{k=0}^{1} \binom{1}{k} f^{1-k}(x)g^{k}(x) $$
        $$ \therefore (f(x)g(x))^1 = f^{1}(x)g(x) + f(x)g^{1}(x) $$
        This is the Leibniz product rule of derivatives. Thus, the base case holds.
      \item Inductive steps:\\
        Assume that the statement is true for some $n \in \mathbb{N}$, i.e.
        $$ (f(x) g(x))^n = \sum_{k=0}^{n} \binom{n}{k} f^{n-k}(x)g^{k}(x) $$

        Differentiating this one more time w.r.t $x$, we get:
        $$ (f(x) g(x))^{n+1} = \frac{d}{dx}\Big[\sum_{k=0}^{n} \binom{n}{k} f^{n-k}(x)g^{k}(x)\Big] $$
        $$ = \frac{d}{dx}\Big[\binom{n}{0}f^{n}(x)g(x) +\binom{n}{1}f^{n-1}(x)g^{1}(x) +\binom{n}{2}f^{n-2}(x)g^{2}(x) \ldots$$
        $$ \ldots \binom{n}{n-2}f^{2}(x)g^{n-2}(x) +\binom{n}{n-1}f^{1}(x)g^{n-1}(x) +\binom{n}{n}f(x)g^{n}(x) \Big]$$

        Expanding the derivative, we get:

        $$ (f(x) g(x))^{n+1}=\binom{n}{0}(f^{n+1}(x)g(x) + f^{n}(x)g^{1}(x))         $$
        $$                 + \binom{n}{1}(f^{n}(x)g^{1}(x) + f^{n-1}(x)g^{2}(x))     $$
        $$                 + \binom{n}{2}(f^{n-1}(x)g^{2}(x) + f^{n-2}(x)g^{3}(x))   $$
        $$                 \vdots                                                    $$
        $$                 + \binom{n}{n-2}(f^{3}(x)g^{n-2}(x) + f^{2}(x)g^{n-1}(x)) $$
        $$                 + \binom{n}{n-1}(f^{2}(x)g^{n-1}(x) + f^{1}(x)g^{n}(x))   $$
        $$                 + \binom{n}{n}(f^{1}(x)g^{n}(x) + f(x)g^{n+1}(x))         $$

        Observe that, in this expansion, every pair of consecutive terms has a common factor. Collecting these factors, we get:
        $$(f(x) g(x))^{n+1} = \binom{n}{0}f^{n+1}(x)g(x) + \sum_{k=1}^{n}\binom{n+1}{k}f^{n+1-k}(x)g^{k}(x) $$
        $$ + \binom{n}{n}f(x)g^{n+1}(x)$$

        Since $\binom{n}{n} = \binom{n}{0} = 1 \forall n \in \mathbb{N}$, we can rewrite the first and the last term of the RHS above in terms of $n + 1$ as follows:
        $$(f(x) g(x))^{n+1} = \binom{n+1}{0}f^{n+1}(x)g(x) + \sum_{k=1}^{n}\binom{n+1}{k}f^{n+1-k}(x)g^{k}(x) $$
        $$ + \binom{n+1}{n+1}f(x)g^{n+1}(x)$$
        $$ = \sum_{k=0}^{n+1}\binom{n+1}{k}f^{n+1-k}(x)g^{k}(x)$$

        Thus we have shown that if
        $$ (f(x) g(x))^n = \sum_{k=0}^{n} \binom{n}{k} f^{n-k}(x)g^{k}(x) $$
        then,
        $$ (f(x) g(x))^{n+1} = \sum_{k=0}^{n+1} \binom{n+1}{k} f^{n+1-k}(x)g^{k}(x) $$

    \end{enumerate}
    Thus, by the principle of mathematical induction, the Leibniz rule holds for any $n \in \mathbb{N}$, provided $f^{n}(x)$ and $g^{n}(x)$ exist.\\
    $\blacksquare$
  \item The cars leave and reach at the same times. Suppose they take $T$ seconds to reach the destination. So, they both leave at $t = 0$ seconds and reach their destination at $t = T$ seconds. Suppose the distance between the stops is $D$ meters, and the instantaneous distance of the cars from the origin is $d_1(t)$ and $d_2(t)$.

    Assuming that the cars cannot teleport, $d_1(t)$ and $d_2(t)$ are both contiuous functions. Their respective velocities, are, then
    $$ v_1(t) = \frac{d_1(t)}{dt} $$
    $$ v_2(t) = \frac{d_2(t)}{dt} $$

    We need to show that there exists some $t^{*} \in [0, T]$ such that $v_1(t^{*}) = v_2(t^{*})$.\\

    Proof by Contradiction:\\
    Let's assume that there does not exist any $t^{*} \in [0, T]$ such that $v_1(t^{*}) = v_2(t^{*})$.\\
    In other words, one of the following cases must hold:
    \begin{enumerate}
      \item $v_1(t) > v_2(t) \forall t \in [0, T]$
      \item $v_1(t) < v_2(t) \forall t \in [0, T]$
    \end{enumerate}

    In the first case, if the instantaneous velocity of the first car is always greater than the instantaneous velocity of the second car, then the first car will have travelled a greater distance than the second car, i.e.\\

    $$ \int_{0}^{T}v_{1}dt > \int_{0}^{T}v_{2}dt $$

    Similarly, for the second case, we get

    $$ \int_{0}^{T}v_{1}dt < \int_{0}^{T}v_{2}dt $$

    But we know that
    $$ \int_{0}^{T}v_{1}dt = \int_{0}^{T}v_{2}dt = D$$

    Thus, if the cars don't have the same instantaneous velocities at least once, they will end up travelling different distances. This is a contradiction.\\

    Therefore the cars must have the same instantaneous velocity at least once in their journey.\\
    $\blacksquare$

\end{enumerate}
\end{document}
