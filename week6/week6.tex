\documentclass[a4paper,12pt]{article} % This dfines the style of your paper
\usepackage[top = 2.5cm, bottom = 2.5cm, left = 2.5cm, right = 2.5cm]{geometry} 
\usepackage[T1]{fontenc}
\usepackage[utf8]{inputenc}
\usepackage{multirow} % Multirow is for tables with multiple rows within one cell.
\usepackage{booktabs} % For even nicer tables.
\usepackage{graphicx} 
\usepackage{setspace}
\usepackage{amsfonts}
\usepackage{amssymb}
\usepackage{amsmath}
\setlength{\parindent}{0in}
\usepackage{float}
\usepackage{fancyhdr}
\pagestyle{fancy} % With this command we can customize the header style.
\fancyhf{} % This makes sure we do not have other information in our header or footer.

\lhead{\footnotesize Graded Assignment: Week 6}% \lhead puts text in the top left corner. \footnotesize sets our font to a smaller size.

\rhead{\footnotesize Jaidev Deshpande (Roll No: 21F1003751)} %<---- Fill in your lastnames.
\cfoot{\footnotesize \thepage} 

\begin{document}

\thispagestyle{empty} % This command disables the header on the first page. 

\begin{tabular}{p{15.5cm}} % This is a simple tabular environment to align your text nicely 
{\large \bf Mathematical Thinking} \\
BS In Data Science \& Applications  \\ Sept 2023 Term  \\ IIT Madras\\
\hline % \hline produces horizontal lines.
\\
\end{tabular} % Our tabular environment ends here.

\vspace*{0.3cm} % Now we want to add some vertical space in between the line and our title.

\begin{center} % Everything within the center environment is centered.
	{\Large \bf Graded Assignment: Week 6} % <---- Don't forget to put in the right number
	\vspace{2mm}
	
        % YOUR NAMES GO HERE
	{\bf Jaidev Deshpande (Roll No: 21F1003751)} % <---- Fill in your names here!
		
\end{center}  

\vspace{0.4cm}

\begin{enumerate}
\item By the Binomial theorem, the polynomial can be expanded as:
	$$ (x + 2y)^7 = \binom{7}{0}x^{7}(2y)^{0} + \binom{7}{1}x^{6}(2y)^{1} +
		\binom{7}{2}x^{5}(2y)^{2} + \dots \binom{7}{0}x^{0}(2y)^{7}
	$$
	We are interested in the third term, i.e.
	$$
		\binom{7}{2}x^{5}(2y)^{2} = \frac{7 \times 6}{2}x^{5}4y^{2}
		= 84x^{5}y^{2}
	$$
		So the coefficient is (b) = 84.

\item The fraction $\frac{225}{157}$ can be expanded with continued fractions as follows:

	$$
	\frac{225}{157} = 1 + \frac{1}{2 + \frac{1}{3 + \frac{1}{4 + \frac{1}{5}}}}
	$$

	Thus, $a = 2, b = 3, c = 4$ and $d = 5$.\\
	$\therefore a + b + c + d = 14$
\item
	\begin{enumerate}
		\item $f(x) = x^{10} + x^9 + x^5 + x + 1$\\
		Computing $x^{10} f(x^{-1})$,
		$$
		x^{10} f(x^{-1}) = x^{10}\Big(\frac{1}{x^{10}} + \frac{1}{x^9} + \frac{1}{x^5} + \frac{1}{x} + 1\Big)
		$$
		$$ = 1 + x + x^5 + x^9 + x^{10}$$
		$$ = f(x) $$
		$\therefore f(x)$ is self-reciprocal.
		\item $f(x) = x^{11} + x^8 + x^5 + x^3 + 1$\\
		Computing $x^{11} f(x^{-1})$,
		$$
		x^{11} f(x^{-1}) = x^{11}\Big(\frac{1}{x^{11}} + \frac{1}{x^8} + \frac{1}{x^5} + \frac{1}{x^3} + 1\Big)
		$$
		$$ = 1 + x^3 + x^6 + x^8 + x^{11}$$
		$$ \neq f(x) $$
		$\therefore f(x)$ is \textbf{not} self-reciprocal.
	\end{enumerate}
\item The degree of the Poincaré polynomial of is the highest number of inversions that are possible in a permutation. For Perm(6), the highest number of inversions possible are 5 + 4 + 3 + 2 + 1 = 15.\\
	$\therefore$ the degree of $f_{6}(x)$ of Perm(6) is \textbf{15}.

\item Consider the Poincaré polynomial $f_{5}(x)$ of Perm(5).
	$$
	f_{5}(x) = (1 + x + x^2 + x^3 + x^4)(1 + x + x^2 + x^3)(1 + x + x^2)(1 + x)
	$$
	The number of permutations that have exactly 8 inversions is the same as the coefficient of $x^8$ in $f_{5}(x)$.\\
	To find the coefficient of $x^8$, we have to pick powers of $x$ from the four terms (not necessarily from all of them) above such that their sum is 8.\\
	For simplicity, let's write the four terms of the Poincaré polynomial in rows:
	$$
	\begin{array}{ccccc}
		x^0 & x^1 & x^2 & x^3 & x^4\\
		x^0 & x^1 & x^2 & x^3\\
		x^0 & x^1 & x^2 & \\
		x^0 & x^1 & & \\
	\end{array}
	$$
	We now need to pick zero or more powers of $x$ from each row such that their sum is 8.\\
	This problem is equivalent to finding the number of ways of writing 8 as a sum of four integers $\{a, b, c, d\} \subset \mathbb{Z}$  such that:

	\begin{enumerate}
		\item $0 \leq a \leq 4$
		\item $0 \leq b \leq 3$
		\item $0 \leq c \leq 2$
		\item $0 \leq b \leq 1$
	\end{enumerate}
	This can be done in nine ways as follows:
	$$8 = 4+3+1+0$$
	$$8 = 4+3+0+1$$
	$$8 = 4+2+2+0$$
	$$8 = 4+2+1+1$$
	$$8 = 4+1+2+1$$
	$$8 = 3+3+2+0$$
	$$8 = 3+3+1+1$$
	$$8 = 3+2+2+1$$
	$$8 = 2+3+2+1$$
	$\therefore$ the number of permutations of $\sigma \in$ Perm(5) that have exactly 8 inversions is \textbf{9}.
\item
	\begin{enumerate}
	\item The student needs to walk exactly 8 blocks east, and 10 blocks north to reach the school.\\
	The number of total paths is
	$$
	\binom{10 + 8}{8} = \frac{18!}{8!10!} = 43758
	$$
	\item Let us assume that the friend also prefers the shortest path, i.e. they never walk west or south. Also, we do \textit{not} care about how the friend reaches the meeting point, i.e. in this problem, we \textbf{do not} count the number of ways in which the friend reaches the meeting point. We care \textit{only} about how many paths the student herself can follow.\\
	Let's denote the meeting point with the pair $(x, y)$, which is located at $x$ blocks east and $y$ blocks north of the student's house.
	Because of the shortest path assumption, $(x, y)$ should lie strictly within the northeast quadrant of the grid. Formally, this means the meeting point is $(x, y)$ such that $4 \leq x \leq 8$ and $5 \leq y \leq 10$.
	Now, the number of paths that pass through $(x, y)$ is the product of the number of paths from the origin to $(x, y)$ and the number of paths from $(x, y)$ to the school. Suppose we denote this with $p_{x,y}$, i.e.\\
	$p_{x, y} := $ \# of paths from the origin to the school which pass through point $(x, y)$ on the grid.\\
	$$\therefore p_{x, y} = \binom{x+y}{x} \times \binom{18-x-y}{8-x}$$
	Thus, the total number of desired paths is:
	$$ \sum_{x=4}^{8}\sum_{y=5}^{10}\binom{x+y}{x} \times \binom{18-x-y}{8-x} $$
	\end{enumerate}

\item
	\begin{enumerate}
	\item $t_1 = 3100210$\\
		$t_2 = 4200210$
	\item $\sigma' = 3427165$
	\item From $t_1$, we see that sorting $\sigma = 3241765$ with simple transpositions should be possible in 7 steps, as follows:
		\begin{enumerate}
			\item swap 3 and 2 to get 2341765
			\item swap 4 and 1 to get 2314765
			\item swap 3 and 1 to get 2134765
			\item swap 2 and 1 to get 1234765
			\item swap 7 and 6 to get 1234675
			\item swap 7 and 5 to get 1234657
			\item swap 6 and 5 to get 1234567
		\end{enumerate}
	\end{enumerate}

\end{enumerate}
\end{document}
